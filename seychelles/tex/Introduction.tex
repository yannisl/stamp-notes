% This is a sample LaTeX input file.  (Version of 12 August 2004.)
%
% A '%' character causes TeX to ignore all remaining text on the line,
% and is used for comments like this one.

\documentclass[10pt,oneside,justified]{tufte-book}     % Specifies the document class
%% decide on fonts
\usepackage{ifxetex}
\ifxetex
  \usepackage{fontspec}
  \defaultfontfeatures{Mapping=tex-text}
  \setmainfont{Minion Pro}
  %\setsansfont{Georgia}
\else
  \usepackage{mathpazo}
  \usepackage[T1]{fontenc}
\fi
%
%\usepackage{url}
%\usepackage{xcolor,filecontents,ragged2e}

\usepackage{graphicx}
\usepackage{amsmath}
\usepackage{makeidx}
\usepackage{xcolor}
\usepackage{hyperref}
%\hypersetup{%
%  colorlinks=true,
%  linkcolor=black,
%  urlcolor=spot,
%  bookmarks=true,
%  bookmarksopen=false,
%  bookmarksnumbered=false,
%  hyperfootnotes=false,
%  plainpages=false,
%  pdfpagelabels=true,
%  pdfpagemode=UseOutlines,
%  pdfview=FitH,
%  pdfstartview=FitH}
                             % The preamble begins here.
\title{An Example Document}  % Declares the document's title.
\author{Dr. Yiannis Lazarides}      % Declares the author's name.

\title{Postal Services\\and\\Postal History\\of\\Slovenia}  % Declares the document's title.
\author{Dr. Yiannis Lazarides}      % Declares the author's name.
                              % Deleting this command produces today's date.
                              % Deleting this command produces today's date.

\def\lorem{Lorem ipsum dolor sit amet, consectetur adipiscing elit. Sed nibh justo, dictum sed cursus ac, lobortis et lacus. Vestibulum vitae justo enim. Quisque laoreet elementum felis, ut sodales arcu viverra a. Sed molestie odio vulputate sem rutrum a sagittis est rutrum. Morbi dapibus hendrerit magna, sit amet commodo massa posuere sit amet. Duis pharetra quam scelerisque est lobortis fringilla. Maecenas venenatis feugiat lectus, vel facilisis odio pharetra quis. Etiam at nisl eros, sit amet suscipit lorem. Lorem ipsum dolor sit amet, consectetur adipiscing elit. Sed augue nunc, ornare eget congue sit amet, laoreet vel augue. Morbi vel justo quis ipsum adipiscing egestas vitae non est. Vivamus ac quam quam. Nullam pharetrainterdum mauris, rutrum pulvinar ligula condimentum id. Donec et blandit lorem. }

\newcommand{\wrapleft}[3][1]{}
\newcommand{\wrapright}[3][1]{}
\newcommand{\ph}[3][1]{%
 \begin{figure}[htbp]
   \includegraphics[#1]{#2}
   \caption{#3}
 \end{figure}}

\def\euro{EU }
\def\pound{\pounds}
\newcommand{\phl}[3][1]{}


\usepackage{hyperref}
\begin{document}             % End of preamble and beginning of text.

\maketitle                   % Produces the title.
\tableofcontents

%\end{document}

<!--{{ }}-->
\#\#Seychelles: Postal history and Stamps
A post office was opened only in 1861 though in 1824 a letter is said to have been "privately carried" to Europe. Until the boat service "Messagerie Maritimes" took over the mail contract in 1866 mail to and from Seychelles could not be transported on a regular basis.

The earliest known letter using the service was dated 3 April 1865.

Colonial reports brought out the extreme communication difficulties though in 1877 the Mauritius Post Office reported that letters to and from Seychelles numbered 14,184(543 were registered).  The Post Office became autonomous from Mauritius on 1 August 1884.

It was in 1893 that the first local post was introduced in Seychelles.

The non-commissioned officers in charge of the Police stations acted as receiving and dispatching officers. The service was closed on 31 July 1894 as it proved too much for them to cope with.

The new Administrator, Sweet Escott, who arrived in the Colony in 1899 saw that the postal service needed improvement and introduced a new Inland Post Service The British India Steam Navigation Company(BISN) had began their contract in August 1895.

The quarantining of ships is said to have undoubtedly affected the mails from time to time.

In 1920, the local post is said to have covered the whole population, and not only Mah\'e and Praslin, including the outer islands and from 1965 the British Indian Ocean Territory.

 

Chapter Two of the book deals with mail by air which was introduced on 23 September 1932 and became effective on 23 February 1938. It could be sent either through Karachi or Nairobi.. It was on 21 June 1939 that the first civil flying machine arrived in Seychelles from Australia. . The Royal Air Force in 1943 established a base in Seychelles and the Catalina flying boats operated communication flights between Seychelles and Mauritius, Madagascar and Kenya. They carried mail in and out but were only authorized to carry properly censored mail for members of the armed forces.

In January 1953 the East African Airways Corporation attempted to introduce the first air mail from Seychelles with a Catalina amphibian. The service was not economic and was withdrawn.

As the United States Air Force (USAF) built its Satellite Tracking Station on Mah\'e in 1963 it introduced the service of a Grumann Albatross amphibian which flew from Mombasa. It started to carry first and second class air mails from 19 August 1964.

The Wilkenair (later Air Kenya) introduced an air mail service with its first flight on 2 May 1970. Then on 4 July 1971 with the opening of the International Airport there was a new weekly direct air mail service between Seychelles and London by a BOAC VC 10.

 

Chapter Three of the book deals with censorship, both civil and military.

During World War l there was no regular censorship of civilian mail to or from Seychelles, it is said. Though all correspondence to what was considered enemy territories was examined by the Governor before dispatch, up to 28 May 1919.

 

During World War ll censorship of mail, telegrams and newspapers started early in Seychelles. The original censors were Government Officers, Police Officers and retired Military Officers supported by staff from Cable and Wireless, for the censorship of telegrams.

 

Chapter Four deals with Mauritius Postage Stamps used in Seychelles as the Victoria General Post Office opened in 1861 was a sub-office of Mauritius and did not have its own postage stamps.

The definitive postage stamps of Queen Victoria came in six issues provisionally in 1893.


\#\#\#Mauritius used in Seychelles
<div style="width:32.5%;float:left;margin-right:10px">
<h4>Z2</h4>
<img src="http://localhost/egypt/seychelles/1513.jpg" style="width:98%;margin-left:10px" />
<p style="font-size:smaller"> 

S.G. \#Z2, Mauritius 1859 6p Blue, a gem used example, featuring four exceptionally large and well balanced 
margins, deep rich color, neat perfect light strike of barred "B64" numeral cancel of Seychelles, 
extremely fine; quite likely the finest existing example of this rare stamp; 1991 BPA certificate. (Image) 	\pound900

</p>
{{auction: Shreves Galleries, January 2009}}
</div>



<div style="width:32.5%;float:left;margin-right:3px">
<h4>Z3</h4>
<img src="http://localhost/egypt/seychelles/1514.jpg" style="width:98%;margin-left:10px" />
<p style="font-size:smaller"> 
S.G. \#Z3, Mauritius 1859 6p Dull purple slate, a lovely four-margined example, 
with rich color and impression on bright paper, identifiable central strike of barred "B64" 
numeral cancel of Seychelles, small thin spot and a light vertical crease at right, still of very fine 
appearance of this important Mauritius used in Seychelles rarity; 2000 BPA certificate.  \pound 2,250

</p>
{{auction: Shreves Galleries, January 2009}}
</div>


<div style="width:32.5%;float:left;margin-right:3px">
<h4>Z4</h4>
<img src="http://localhost/egypt/seychelles/Z3.jpg" style="width:98%;margin-left:10px" />
<p style="font-size:smaller"> 
Sale 5006 Lot 1157

Seychelles
Mauritius Used In Seychelles
The "B64" Obliterator
1859-61 Imperforate Britannia Values
1/- vermilion with good to large margins nearly all round, a fine upright virtually complete strike. An attractive example of this rare stamp. B.P.A. Certificate (2004). S.G. Z4, \pound1,300. Photo
Estimate \pound 600-700 
</p>
{{Spinks: Shreves Galleries, January 2009}}
</div>

<div style="width:32.5%;float:left;margin-right:3px">
<h4>Z12</h4>
<img src="http://localhost/egypt/seychelles/1516.jpg" style="width:98%;margin-left:10px" />
<p style="font-size:smaller"> 
S.G. \#Z12, Mauritius 1860-63 1/- Green, attractive sound example with identifiable central (inverted) strike of barred 
"B64" numeral cancel of Seychelles, fine and rare; 1978 RPS certificate. (Image) 	\pound800 
</p>
{{auction: Shreves Galleries, January 2009}}
</div>

<div style="clear:both"></div>
<div style="width:32.5%;float:left;">
<h4>Z13</h4>
<img src="http://localhost/egypt/seychelles/1517.jpg" style="width:98%;margin-left:10px" />
<p style="font-size:smaller"> 
S.G. \#Z13, Mauritius 1862 6p Slate, attractive sound example with excellent central (inverted) 
strike of barred "B64" numeral cancel of Seychelles, fine and rare. (Image) 	\pound800 
</p>
{{auction: Shreves Galleries, January 2009}}
</div>
<div style="clear:both"></div>
<div style="width:32.5%;float:left;">
<h4>Z13</h4>
<img src="http://localhost/egypt/seychelles/1521.jpg" style="width:98%;margin-left:10px" />
<p style="font-size:smaller"> 
S.G. \#ZR1, Mauritius 1889 4c Lilac "Inland Revenue" postal fiscal, choice example with superb complete strike of barred "B64" numeral cancel of Seychelles, very fine and exceedingly rare; ex-Cdr. M. Burnett. (Image) 	\pound1,200

</p>
{{auction: Shreves Galleries, January 2009}}
</div>
<div style="clear:both"></div>



          
\end{document}