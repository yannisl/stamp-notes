Cape of Good Hope - Postal History

\section{Travelling Post Offices}
\section{The No Name Travelling Post Offices}

\ph[width = .35\textwidth]{../cape-of-good-hope/TPO_Images/tpo_up.png}{ 
T.P.O. marks can be identified by 
measuring the diameter of the circle the distance between the 
letters 'U' and 'P' and the width of the letter 'U'.
}



	 


The first T.P.O to operate was started in 1882 and originally operated 
between Cape Town and Beaufort West. 
This was later extended to Victoria West Road (later named Hutchinson) 
and subsequently to De Aar and beyond. 

These first marks had no special identification marks and they were simply 
denoted T.P.O. UP, T.P.O.DOWN, 
TRAVELLING PO UP or TRAVELLING PO DOWN. These marks can be found used up to 1896.

To identify these marks properly both the illustration and the 
description should be consulted.

Hagen \& Naylor describe thirteen date stamps used for this purpose.
T.P.O. marks can be identified by measuring the diameter of the 
circle the distance between the letters 'U' and 'P' and the width of the letter 'U'.

In about 1891 the TPO between Cape Town and De Aar was renamed
Western TPO, reflecting its route of operation 
on the Western Section of the CGR. 
Well after the new name has been adopted, the use of No Name continued 
until at least May 1896 (Hagen and Naylor). 
The service may have operated also on the route De Aar Naauwpoort and 
after the rail bridge over the Orange River was opened in 1890, as far 
north as Norvalspont.

Postmarks with the 'UP' variety can generally be identified by 
measuring the width of the letter 'U' and the distance between 
the letters 'U' and 'P' in 'UP'. 'Down' varieties can be identified 
by the dimensioning of the space between the 'D' and 'N' of 'DOWN'. 
The 'TRAVELLING PO' marks are easily identifiable by their wording 
and no special varieties exist.

Based on this identification method Hagen\& Naylor identified 
thirteen postmarks as described in the following table.

         