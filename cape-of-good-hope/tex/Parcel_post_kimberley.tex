\section{Griqualand - Kimberley}

What appears to be a new discovery in Germany (they seem to have collected 
the best stamps) is the Griqualand block shown below. 
BONC 227 is extremely scarce with only a few copies recorded. 
There are also some known fakes.

\ph[width = .70\textwidth]{../cape-of-good-hope/griqualand-block.jpg}{
5 Sh, block of nine, three stamps as well in the landscape format, 
postmark "227" KIMBERLEY. Rare large unit (street), 
certificate with photograph Knopke. (Catalog value: 1 530)
euro600
}





\ph[width = .50\textwidth]{../cape-of-good-hope/griqualand-block-certificate.jpg}{ 
Certificate from Dr. Knopke for the block.
}

One can see clearly that the handstamp was deteriorating and 
perhaps this is the reason why it is so scarce as they 
probably threw it out. Guessing from the value of the stamps 
this was probably some heavy parcel, and probably heavily insured. 
There are totally nine stamps of 5 /- each, which was a lot 
of money in those days. One would like to think that it 
contained diamonds destined for overseas. 

As we don't know the year it was posted, one can assume 
the cheapest rate possible at the time which was 9d per lb. 
This was then around a 60lb parcel (27 kg). Kimberley at 
the time was renown for its diamond mining and one can 
only guess that it contained uncut diamonds.

I bid the maximum I could afford hoping that nobody in
the room would be interested in the lot. The auction 
is in September and I am hoping to win it at around euro350. 
Anything less than this would be a great bargain. the stamps 
alone are worth five times that. 

	 
                                                  