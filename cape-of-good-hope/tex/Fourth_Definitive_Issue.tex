\subsection{The Fourth Definitive Issue
} \heading{(1882-1883)}


\ph[width = .89\textwidth]{../cape-of-good-hope/82-83-die-proof.jpg}{
882-83 DIE PROOF for 2d (SG 42) in black on glazed card marked BEFORE HARDENING and 
dated 12 JUN 82. VF, scarce. ...ebph/019	
Price: \$ 510
}


Early in 1882 a new type of paper was introduced for the various colonies. This paper was watermarked with a crown and C A (Crown Agents) in place of the Crown CC (Crown Colony) which had been in use since about 1863.

The first stamp to be issued on this paper was the Three Pence value, which was issued in July 1882. For the Cape this period was of comparatively short duration as it would have paper with its own watermark the cabled Anchor in future.

 


 
\subsection{The One Penny Stamp}

(Issued August 1882)

This was issued in August 1882 by which time stocks of the previous issues were depleted. It is in all respects similar to its predecessor with the exception of the change of watermark. A considerable number of printings were made resulting in the production of the stamps in a series of readily graduated shades.

 


 
\subsection{The Six Pence Mauve}

(Issued August 1882)

This was also issued in August 1882. It is also similar to previous issues with the exception of the watermark. This stamp retained the thin outer frame surrounding the design.

 

 

 

 


 
The Half Penny Black(Issued September 1882)

These stamps were ordered in April but did not reach Cape Town in time to prevent the necessity for the local Half-penny provisional which was issued in July. Grey-black was employed again although some of the printings were made in a shade closely approaching black.

The use of the Crown CA paper did not last long in the Cape of Good Hope. It was soon followed by other issues on Cape of Good hope's own 'Cabled Anchor' watermarked paper.

 

 


 

 

 

             