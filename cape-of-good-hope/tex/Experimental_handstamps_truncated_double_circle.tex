\section{The Truncated Double Circle Datestamp of 1890}

\ph[width = .90\textwidth]{../cape-of-good-hope/SQUARE-CIRCLE/TRUNCATED/TDC16dec.JPG}{
Earliest Known Cover 1890 Cover from Cape Town  to Worcester, bearing 
seated Hope 1d red and handstamped Truncated Double Circle  
(Goldblatt TDC1) datestamp dated 6 Dec  90, Control letter O. 
backstamped Worcester cds DEC 7 90.
}

The Truncated Double Circle datestamp (Goldblatt  TDC 1 and TDC 2) was 
issued to the Cape Town General Post Office in 1890, with a variant 
to the Graaf-Reinet post office. The design is based on an inner and 
outer circle, the inner divided by two heavy horizontal bars into 
three sections an upper and lower segment, with a central area 
containing the date.

The Truncated Double Circle datestamp issued to the Cape Town G.P.O. 
(TDC 1) has a control letter in the upper segment, whereas the lower
is blank. The outer circle diameter measures 29 mm and is broken at
the bottom. The inner circle measures 18 mm.

\ph[width = .60\textwidth]{../cape-of-good-hope/SQUARE-CIRCLE/TRUNCATED/TDC.jpg}{
The Truncated Double Circle datestamp issued only to the Cape 
Town G.P.O. and Graaf-Reinet.
}

TDC 2, in use in Graaf-Reinet, differs from the Cape Town strike in 
that the letters C.G.H. follow 
the name of the town and there is no cross at the break of the outer circle. 
A small star is located in the lower segment of the inner circle.

The Graaf-Reinet datestamp is a bit difficult to find, but is rather
easier to get one for Cape Town.

\ph[width = .80\textwidth]{../cape-of-good-hope/truncated.jpg}{CAPE OF GOOD HOPE SOUTH AFRICA 1891 POSTAL STATIONERY CARD CAPETOWN- SWELLENDAM \$5.99}

\ph[width = .60\textwidth]{../cape-of-good-hope/truncated-01.jpg}{The back of the card.}

I have not seen the datestamp being used for backstamping, which  continued using  circular datestamps 



 





 

                   