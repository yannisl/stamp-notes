\chapter{Postal Stationery}

\section{First Post Card 1d., red.}

On March ist, 1878, post cards were first introduced.
1d., red, 4 7/8 x 2 1/8, on buff thin cardboard.
Printed locally by Messrs. Saul Solomon \& Co., by lithography.
Several cards were printed in black, as proofs, and some got mixed with
the others, and were sold in the packets to the public; and are therefore
known postally used. \footnote{Note on the Cape Stamps.
COMPILED FROM THE OFFICIAL AND OTHER RECORDS AT CAPE TOWN. BY CAPT. NORRIS-NEWMAN.
(1894). \textit{London Philatelist}, VOL. V. September, 1896. No. 57. }

These cards were for town delivery and suburbs only, or to places
between which there were two posts a day; if sent further, they had to
have additional stamps affixed. 

\ph[width = .80\textwidth]{../cape-of-good-hope/postal-card-1878.jpg}{}

There is no record of how many were
issued, but they were in use up to 1884, when the last was sold out.

\section{Second Post Card.}

1d.

1st July, 1882. id., colour red-brown, on thick white card. Oval stamp,
4rf x 2l  zzCOMMANDzz, or 171 x 74 mil.

Printed by De La Rue, singly and in quarter reams of 120 sheets, 42
cards to each sheet.

There have been nine printings up to 1894  = 7,242,720 cards, being
several years' supply.

These were for intercolonial use---Orange Free State, British Bechuanaland,
Basutoland, Transkie, S.A. Republic, Natal. The card rate was
reduced to \half d. in 1889, for the Cape only.
1000 of these cards were overprinted and supplied to the British South Africa
Company on 15th Oct., 1892.

396 of these cards were overprinted and supplied to the British South Africa
Company in January, 1893.

\section{Third Post Card}.

\half d.

1st January, 1889, a reduction of inland rate was made to \half d., and a card
issued.

Light brown, square stamp, on white thick cardboard. Printed by De
La Rue, singly and in quarter reams, in four printings = 777,120; but,
owing to the colour being too much like that of the 1d. card of 1882,
it was changed to green on the 29th Sept, 1891, and these latter are
still in use. Of these there have been up to date four printings = 5,811,840
cards.

These cards circulated over all the Colony and its dependencies; and
in September, 1892, to the Orange Free State also.

\section{Fourth Post Card}

\subsection{1 1/2d.}

1st May, 1890. 1 1/2 d. card, grey-green on light buff, thin card, with square
stamp, and inscriptions in English and French ; was printed by De La Rue,
for circulation between the Cape and Great Britain, abroad only with an
extra 1d. stamp.

There were four printings = 679,800, and it was withdrawn on 1st
January, 1892 ; and used thereafter only for foreign countries from ist Sept.,
1892, under agreement with U.P.U.
Of these cards,

3000 were overprinted and supplied for use in British Bechuanaland, Nov., 1892.

1000 were overprinted and supplied for use in British South Africa Co., i5th Oct., 1892.

396 were overprinted and supplied for use in Jan-> l893-

NOTE.---The first printing of these cards = 246,840, was sent out from England
in June, 1884, in readiness for the Cape coming under the Postal Union; but negotiations
falling through, they were kept until 1890, as above.

\section{Fifth Post Card.}

\half d. x \half d.

ist July, 1892, a reply-paid post card of id., in brown on thick buff,
140x89, was issued, printed by De La Rue---one printing only = 48,720---for
use to Great Britain only. From the ist Sept., 1892, it was also allowed to
be sent to foreign countries, provided an extra stamp of \half d. was attached
to each half.

Of these cards,

1200 were overprinted and supplied to British Bechuanaland, 5th Oct., 1892.
1000 „ „ British South Africa Company, 15th Oct., 1892.
396 ,, „ „ „ Jan., 1893.

\section{Sixth Post Card.}

\half d.x \half d.

April, 1893. Reply-paid inland card issued, \half d. x \half d., brown on buff, thick
card ; for use in South Africa only. Printed by De La Rue, in two printings
of 48,000 each = 96,000. The 1 \half d. 1\half d. were under order when Captain Norris-Newman was in Cape Town.

\section{Embossed Envelopes}

1st June, 1892, 1d., oval, on white thick laid paper, in two sizes, made by
De La Rue, was issued.

\begin{tabular}{llr}
Size A  &4 3/4 x 3 11/16 &Total printed to 1894, 254,720\\
Size  B &5 13/16 x 3 9/16 &Total printed to 1894  216,000\\
\end{tabular}

And the \half d. and 2 \half d. have came out in 1894.


\section{Registered Letter Envelopes}

Up to 1881 the fee for registration had always been 8d., prepayable by
ordinary stamps; but on the 15th July, 1881, the fee was reduced to 4d., and
Messrs. Mc Corquodale & Co. supplied the first lot of registered envelopes
with 4d., blue, embossed stamps on the flap, in five sizes. This envelope was
on thick white linen-faced wove, and bore their imprint on the inside, where
the flap covered the back part of the envelopes. Supplied as follows:

\begin{tabular}{llr}
Size  F  &5 1/4 x3 1/4  & 21,552\\
 G    & 6x3 3/4       &21,120.\\
 H   & 8 x 5     & 5,016\\
 I   &10x7      & 5,352\\
 K   &11 1/2 x 6      & 10,632\\
\end{tabular}

Since 1882 the supplies have come from De La Rue as above, in number,
to 1894:

\begin{tabular}{llr}
Size F   &5 1/4 x3 1/4     &60,402\\
 G      & 6x3 3/4     &60,216\\
 H    & 8 x 5   &12,196\\
 I     & 10x7     & 3,912\\
 K     & 11 1/2 x6     & 7,752\\
\end{tabular}

A supply of each of the latter was sent to British Bechuanaland, number
not recorded, surcharged in two lines, small letters BRITISH BECHUANALAND; and
also other similar ones for use in the northern Protectorate, surcharged in
two straight lines, m large capitals, BECHUANALAND PROTECTORATE. These latter are
decidedly rare.

The Mc Corquodale & Co. registration enveloped used are difficult to find, especially the larger sizes, as well as the early De La Rue envelopes. 


\section{Wrappers}

Dec. 1st, 1881. \half d., grey-green on buff, 12x5, with inscription and
gummed flap; made by De La Rue, singly, in cut sheets, and quarter
reams of uncut; and ungummed bands of 120 sheets, of 14 wrappers on
each.

Nov., 1892. \half d., colour changed to dark green; similar paper and
details.

Nov., 1892. 1d., red-brown on buff, 12 x 5 ; De La Rue; similar paper
and details.

11th Aug., 1892. Issue of book wrappers, 15x7; parcel of 167 sheets,
of 10 wrappers on each.

4th October, 1892. 1\half d., grey on thin whitish paper, 15x7. De La
Rue. These were sold in quarter reams of 120 sheets, of 10 wrappers each.
Some of each of the above were surcharged for use in British Bechuanaland,
1886-1893, but no numbers recorded.

\ph[width = .80\textwidth]{../cape-of-good-hope/british-bechuanaland-wrapper.jpg}{ }








