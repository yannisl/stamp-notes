\section{Cape of Good Hope - Postal History
} 
\subsection{The Dutch Period in the Cape of Good Hope 
} 
\ph[width = .90\textwidth]{../cape-of-good-hope/voc-handstamp_clip_image002.jpg}{
C 1794 Wrapper addressed to the Cape of Good  
Hope in German showing a single strike of the VOC 6 stuiver handstamp
  (Goldblatt  VOC 3) (photo-cert.: PFSA) 
  }
\subsection{The VOC Handstamp
}             
A proclamation signed by the Acting Governor,  
<a href="personalities/Johannes-Izaac-Rhenius.html">Johan Isaac Rhenius</a> 
on 2nd March 1792 formed the basis of the  
establishment of a postal service at the Cape of Good Hope Colony. 
The post office dealt exclusively  with mail to and from overseas.
(see establishment of <a href="Prestamp-FirstPostOffice.html">First Cape of Good Hope post office ). </a>

A lot of this correspondence was official and addressed to Cape of
 Good Government<a href="voc-capegovernors.html"> Governors</a> and officials.
            
            
\ph[width = .90\textwidth]{../cape-of-good-hope/VOC/voc2.jpg}{ }

The  first postal marking associated with the Cape is the VOC handstamp. 
It consists of a  circle containing the monogram Vereenigde (Nederlandsche Geoctroyeerde)  Oos-Indische Companie; the United Netherlands Chartered East India Company.
Covers addressed to the Cape of Good Hope and stamped with the VOC 
handstamp are scarce.  The collector should be careful as forgeries are known. 
These covers are best bought with a certificate. 

\subsection{Postal History Covers with Two Strikes of the VOC handstamp&nbsp;&nbsp;&nbsp; (12 stuiver rate) 
} 
\ph[width = .90\textwidth]{../cape-of-good-hope/VOC/VOC-cover-double-rate.jpg}{
C  1795 Wrapper addressed to de Heer Hendrik  
Justus de Wet Deutche Consul de Caap de Goede Hoop. Marked by the ship 
Beshout Capt Chr. Martens. At the right hand corner  
there is a talismanic inscription D.g.g.(die god geleide entrusted to God).  
Cover bears two fine strikes of the VOC six stuiver handstamp. (Goldblatt VOC-3)
}

<p style="font-family:arial;font-size:12px;float:left">Provenance: Photo-certified PFSA (ex-Goldblatt Collection) </p><div style="clear:both"></div>            



              
These covers are more scarce than those stamped with only one VOC handstamp.

<strong>Three Strikes (18 stuiver rate) </strong>

Covers with three strikes (18 stuiver rate) are even scarcer and must be considered of the utmost rarity. One such cover  is shown on this page by kind permission. </p>
           
Only one cover with a VOC 2 guilder handstamp addressed to the Cape of Good Hope has been reported in the literature.(See image below). This was received in the Cape on 16 January 1793. </p>

\ph[width = .95\textwidth]{../cape-of-good-hope/VOC/voc-triple.jpg}{
Cover to the Cape of Good Hope struck with three strikes of the VOC
handstamp. Talismanic inscription at lower left hand corner. 
}

\subsection{Was the VOC handstamp a Cape of Good Hope handstamp}

To my knowledge all known covers are addressed to the Cape of Good Hope 
and at best the VOC was an arrival stamp. However, I share Goldblatt's 
view that it is highly unlikely that so many different arrivals 
stamps would have been issued to the Cape, especially given the 
fact that these early mails were limited in number. 
           
\ph[width = .80\textwidth]{../cape-of-good-hope/VOC/voc2.jpg}{
VOC two guilder handstamp
Cover from the Netherlands with the Dutch two guilder VOC handstamp. 
The VOC strike was probably applied on Departure. 
The letter was received at the Cape of Good Hope on 16 January 1793. 
At bottom left there is a talismanic inscription 
'Met Heer en Vrind Die god geleijde in Salvo', i.e. 
Per gentleman and friend entrusted to God in Heaven).}  
 
            
\subsection{VOC Handstamp Classification 
} 
VOC handstamps are normally 
classified based on the angle from the vertical of the 
Number 6 (as per Goldblatt). 
This is still an area where further research can lit some light. 
(See VOC handstamp classification and examples below).


\phl[width = .22\textwidth]{../cape-of-good-hope/VOC/VOC 1.jpg}{
Diameter of circle 18 mm. Figure 6 is 4.5 mm 
from the base to apex and the angle of 
inclination is virtually nil (i.e vertical)
}
\phl[width = .22\textwidth]{../cape-of-good-hope/VOC/VOC 2.jpg}{
Diameter of circle 19 mm. Figure 6 is .5 mm from the 
base to apex and the angle of inclination is approximately 40 degrees.} 
\phl[width = .22\textwidth]{../cape-of-good-hope/VOC/VOC 3.jpg}{
The Diameter of the is circle 18 mm. Figure 6 is 5.5 mm from the base 
to apex and the angle of inclination is approximately 10 degrees. 
The base of the figure virtually touches the circle of the strike. 
This is the major identifying feature of this strike (see photos above) 
Further, the righthand serif to the letter V is inclined at approximately 
15 degrees to the horizontal. 
}
\phl[width = .22\textwidth]{../cape-of-good-hope/VOC/VOC 4.jpg}{
Diameter of circle 18 mm. Figure 6 is 4.5 mm from the 
base to apex and the angle of inclination is approximately 
30 degrees. The intersection of the upper portion of the letter 
C and the righthand arm of the letter V is lower than 
in other types: 3 mm from the top of the serif of the V.}


\div

\subsection{Researching these Covers
} 
Two archive websites provide valuable information on the voyages and passages to the Cape of Good Hope:

http://vocopvarenden.nationaalarchief.nl/detail.aspx?ID=1055256

http://www.historici.nl/Onderzoek/Projecten/DAS/detailVoyage/99080

They have excellent search facilities and although not possible to trace passengers at least some of the VOC settlers and officials can be traced. Based on the dates of the covers one can trace these early ships.








                                           