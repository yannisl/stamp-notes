\#\# Fiume

After World War I, the city of Fiume (now Rijeka, Croatia) was claimed by both the Kingdom of Serbs, Croats and Slovenes and Italy. While its status was unresolved, its postal system was operated by a variety of occupation forces and local governments.



\ph[width=.70\textwidth]{../fiume/fiume-hungary.jpg}{FIUME 1919 - INTERESTING EXPRESS w/ BL.OF 10 H373 \$20. }






The first Italian postage stamps for Fiume were issued 2 December 1918. They were produced by overprinting "FIUME" on the contemporary stamps of Hungary. Both handstamping and printing presses were used. In January 1919, two postage due and a savings bank stamp were surcharged as well. These stamps even the most common values were extensively forged. Serious collectors will require close examination of all stamps.



January also saw the first appearance of an issue produced specifically for Fiume. It consisted of 17 values, ranging from 2 centesmi to 10 corona, and used four designs: a figure representing "Italy", the town clock tower with an Italian flag hanging from it, an allegory of "Revolution", and a sailor raising the Italian flag. The first printings were inscribed just "FIUME", while in July they were redesigned with the inscription "POSTA FIUME", along with other minor changes. Meanwhile, a set of 12 semi-postal stamps was issued 18 May, commemorating the 200th day of peace since the end of the war.
Later in 1919 the higher values were surcharged with lower values, and the semi-postals were overprinted "Valore globale" for use as regular stamps.

\ph[width=.80\textwidth]{../fiume/1919-censored-cover.jpg}{FIUME. 1919. Censored envelope to France bearing 5 c yellow green (Yvert 34) and 25 c blue (Yvert 38) with military cachet COMISSION CENSURA TELEGRAFIC MILITAIRE FIUME with circular 2 censor cachet and VERIFICATO PER CENSURA label. Nice item.  }



Gabriele d'Annunzio, 1920
On 12 September, 1920, the 1st anniversary of the city's takeover by the forces of Gabriele d'Annunzio, the city government issued a series of 14 values featuring a portrait bust of d'Annunzio, intended for regular use, and a set of four with various allegorical designs, intended for the use of the legionnaires on that day only.
On 18 November, the four commemoratives of 12 September were overprinted "ARBE" and "VEGLIA", marking the occupation of the islands of Arbe and Veglia, and on 20 November, more were overprinted "Reggenza / Italiana / del / Carnaro", and with new values.

In January 1921, Italian troops put an end to d'Annunzio's rule, and the subsequent provisional government overprinted the d'Annunzio heads with "Governo / Provvisorio".



On 24 April 1921, the 1st constituent assembly overprinted the semi-postals of 1919 with "24 - IV - 1921" and "Costituente Fiumana". 

The following year the 2nd assembly added a "1922" to the overprints.

\ph[width=.85\textwidth]{../fiume/1922-cover.jpg}{FIUME. Sg.163, 174. 1922 (Feb 20). Registered cover to PARIS bearing Provisional Government 5c. green and 2l. deep claret tied by FIUME cds's. TRIESTE transit on reverse (Feb 22). }


On 23 March, 1923 a new issue put an end to the flurry of overprints. Its 12 values, inscribed "Posta di Fiume", used four designs, a Venetian sailing ship, a Roman arch, St. Vitus, and a rostral column, all printed over a buff-colored background. After the Treaty of Rome assigned Fiume to Italy (27 January), these stamps were overprinted "REGNO / D'ITALIA" (22 February) and then "ANNESSIONE / ALL'ITALIA" (1 March). Subsequently Fiume used the stamps of Italy.

\ph[width=.85\textwidth]{../fiume/1920-postcard.jpg}{FIUME 1920 - NICE PPC w/ OVPR REGGANZA ITAL. DEL CARNARO H400 \$10.50. }


                               