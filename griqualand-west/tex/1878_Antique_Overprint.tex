\chapter{1878 Antique Overprint}    

By the middle of 1878 a new supply of stamps was required to be overprinted. As the previous type was clearly wearing out and new font was used. This time a smaller 'G' was used in a type known as 'antique face'. Both roman as well as italics were used in the setting of 120. Holmes report that he had in his collection full panes of the \halfd, so it is easier to identify the positioning of the different fonts.  He reported<sup>1</sup> that the left-hand pane was made up entirely of roman G's, but the right-hand pane consisted of a mixture of both, with the italic lettering predominating.
The upright roman lettering can be found in positions 6, 23, 36, 42, 48, 54 and 60. \textit{Se-tenant} pairs showing both types are rare.

The overprinting was in red or black or both on some varieties. The values are as shown in the table below:

\begin{tabular}{llllll}
  &\textit{Red overprint}   &$\frac{1}{2}$d. grey-black &&&\\
  &                         &4d. blue (without frame line).&&&\\
  &\textit{Black overprint} &$\frac{1}{2}$d. grey-black.&&&\\
  &                         &1d. carmine-rose.&&&\\
  &                         &4d. blue (with frame line).&&&\\
  &                         &4d. blue (without frame line).&&&\\
  &                         &6d. violet.&&&\\
  &Double overprint (black and red)        &&&&\\
  &                         &$\frac{1}{2}$d. grey-black.&&&\\
  &                         &1d. carmine-rose.&&&\\
\end{tabular}


\ph[width = .80\textwidth]{../griqualand-west/AC388.jpg}{AC388	1878 watermark Crown CC, \halfd grey-black. Vertical interpanneau block of eight, the upper four stamps showing overprint 'G' type 15 (upright) and the lower four type 16 (inverted), clearly demonstrating that the setting of 120 was sometimes inverted on the lower two panes. The margin also showing 'CROWN' of watermark. Marginal crease barely affects one, otherwise very fine or unmounted mint. Scarce and attractive. SG 14, 15a	\pound475 }

\subsubsection{Double overprints}

\ref{inva} and \ref{invb} show specimens with the overprint double the one in red and the other in black. Holmes thought that these could have come out of a trial run or printed in the wrong colour and then overprinted once again with black ink to correct the error.

\ph[width = .50\textwidth]{../griqualand-west/AC632.jpg}{AC632	1878 1d carmine-red overprinted 'G' (Type 15). With variety OVERPRINT DOUBLE, BOTH INVERTED WITH ONE IN RED. Thinned and with perforation faults, otherwise fine mint and of reasonable appearance. SG 16d	\pound25 \label{invb} }

\ph[width = .50\textwidth]{../griqualand-west/AC631.jpg}{AC631	1878 1d carmine-red overprinted 'G' (Type 16). With variety OVERPRINT DOUBLE, BOTH INVERTED WITH ONE IN RED. Hinge remnants, otherwise fine mint. Scarce. SG 17d \pound110.\label{inva}}





1 \LP{Holmes H.R.}{The Postage Stamps of Griqualand West, Part III}{January 1963}{71:841, p.1 LP841.pdf}                                          